\documentclass{jsarticle}
  \title{通常最小二乗法(Ordinary Least Squares: OLS)}
  \author{T.Chiba}
  \usepackage{latexsym}
  \usepackage[dvipdfmx]{graphicx}
  \usepackage{array, booktabs}
  \usepackage{float}
  \usepackage{moreverb}
  \usepackage{here}   %図の挿入場所を「ここ」で固定 H
  \usepackage{fancyvrb}
  %\usepackage{amsmath, amssymb,theorem}
  \usepackage{ascmac}
  \usepackage{geometry}%余白とか縮小するやつ
  \usepackage{bm}      %ベクトル
  \usepackage{cancel}  %文字の打ち消し線
  \usepackage{url}     %URL
  \usepackage{mathrsfs}%花文字 mathscr{ABCD...}
  \usepackage{comment} %複数行にわたるコメントアウトの実現 \begin{comment}~\end{commnet}
  \usepackage{mathtools}
  \usepackage{newtxtext,newtxmath, theorem} % フォント
  
  \geometry{left=30mm,right=30mm,top=25mm,bottom=25mm}
  
  \theoremstyle{plain}
   \newtheorem{theo}{定理}[section]
   \renewcommand{\thetheo}{}
   \newtheorem{defi}{定義}[section]
   \renewcommand{\thedefi}{}
   \newtheorem{lemm}{補題}[section]
   \renewcommand{\thelemm}{}
   \newtheorem{Proof}{証明}[section]
   \renewcommand{\theProof}{}
   \newtheorem{exam}{例}[section]
   \renewcommand{\theexam}{}
  \def\qed{\hfill $\Box$} %証明終わりの記号
  
  % 式番号に節番号を付加 (section.subsection)
  %\def\theequation{\thesection.\arabic{equation}}
  %\makeatletter
  %\@addtoreset{equation}{subsection}
  %\makeatother
  
  % section.subsection.subsection
  %\makeatletter
  % \renewcommand{\theequation}{%
  %   \thesubsection.\arabic{equation}}
  %  \@addtoreset{equation}{subsection}
  %\makeatother

  \begin{document}
  % タイトルの表示
  \maketitle
  \begin{abstract}
    最小二乗法についてみていく。
  \end{abstract}
  \section{最小二乗法(1変量)}
    本題に入る前に、リハビリもかねて1変量の最小二乗法を確認する。
    線形回帰モデルは下記のような構造をしていると考えられる。
  \begin{equation}
    y = \alpha + \beta x + \varepsilon
  \end{equation}
    $x, y$ についての観測値 $\{(x_i, y_i) | i = 1,2,\ldots,n\}$ がそれぞれ次のような構造を持っていると仮定する。
  \begin{equation}
    y_i = \alpha + \beta x_i + \varepsilon_i, i = 1,\ldots,n
  \end{equation}
  誤差 $\varepsilon$ の平方和を考える。
  \begin{equation}
    Q(\alpha, \beta) = \sum_{i=1}^n \varepsilon_i^2 = \sum_{i=1}^n \{y_i - (\alpha + \beta x_i)^2 \}
  \end{equation}
  これを。偏差平方和と呼ぶ。$Q(\alpha, \beta)$ を最小にするような $\alpha = \hat{\alpha}, \beta = \hat{\beta}$ を最小二乗推定量(Least Squares Estimator: LSE)と呼ぶ。

  \begin{eqnarray}
    \frac{\partial Q(\alpha, \beta)}{\partial \alpha} &=& 0 \\
    \frac{\partial Q(\alpha, \beta)}{\partial \beta} &=& 0
  \end{eqnarray}
  を満たすような、$\alpha, \beta$ が偏差平方和を最小にする値である。
  
  \appendix
  \section{}
  \subsection{2変数関数の極値}

  
  \begin{thebibliography}{9}
    \bibitem{} 鈴木 武, 山田 作太郎:『数理統計学 -基礎から学ぶデータ解析』, 内田老鶴圃, 1996
    \bibitem{Inagaki} 稲垣宣生:『数学シリーズ 数理統計学(改訂版)』, 裳華房, 2003.
  \end{thebibliography}
\end{document}
